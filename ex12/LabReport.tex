\documentclass{article}

\title{COMP26120 Lab 12: Background}
\author{Anca Radulian}

\begin{document}
\maketitle

% PART 1 %%%%%%%%%%%%%%%%%%%%%%%%%%%%%%%%%%%%%%%%%%%%%%%%%%%%%%%%%%%%%%%%%%%%%%

\section{The small-world hypothesis}
\label{sec:small world}
% Here give your statement of the small-world hypothesis and how you
% are going to test it.

The small world hypothesis states that between any pair of nodes(persons) there
is a short path of around 6 six nodes(sequence of acquaintances).

To test the hypothesis, calculate the distances between all the nodes to check
how many of them are less than 6. Then compute a percentage to verify how close
is the graph to posses the small world hypothesis.

\section{Complexity Arguments}
\label{sec:complexity}
% Write down the complexity of Dijkstra's algorithm and of Floyd's algorithm.
% Explain why, for these graphs, Dijkstra's algorithm is more efficient.

Dijkstra's algorithm calculates the shorthest path between a source node and all
the other nodes, whereas Floyd's algorithm calculates all the shortest paths
between all the nodes.

Complexities(E = edges, N = nodes):
    O(E logN) - Dijkstra
    O(N^3)    - Floyd

Is better to use Dijkstra' algorithm for these kind of graphs due to its
complexity and because we do not have edges with negative weights. Also Dijkstra
works better on adjacency lists.

\section{Part 1 results}
\label{sec:part1}
% Give the results of part one experiments.

For the Caltech file is it possible to run the Dijkstra algorithm for all the
nodes since there are only 769 nodes. We can observe that there are just a few
nodes that do not have paths with less than 6 nodes.

For the Oklahoma file since there are so many nodes it takes too long to run the
algorithm for all the nodes. I have chosen to run the algorithm with the first
node as a source. The results obtained suggest that there are 5 nodes that cannot
be reached from node 1 in less than 6 steps.
If we run the algorithm starting with isolated nodes such as 9619, the percentage
will be much smaller due to lack of relations with other nodes.

% PART 2 %%%%%%%%%%%%%%%%%%%%%%%%%%%%%%%%%%%%%%%%%%%%%%%%%%%%%%%%%%%%%%%%%%%%%%

\section{Part 2 complexity analysis}
\label{sec:complexity2}
% Give the complexity of the heuristic route finder.

The complexities of the heuristic route finder are:
  - from one node to another is O(N)
  - from one node to all the other nodes is O(N^2)
  - for all paths for all the nodes is O(N^3)

\section{Part 2 results}
\label{sec:part2}
% Give the results of part two experiments.

The result for running the heuristic path finder(based on out degree) to find
all the paths for all the nodes is considerable smaller than the one obtained
using Dijkstra.

For the Caltech file is around 68\% which means that is not an efficient way to
search for all the paths.

For the Oklahoma file, the method is called having as source the node 1 and it
calculates all the paths from this node. The result is around 34\%.

\end{document}
